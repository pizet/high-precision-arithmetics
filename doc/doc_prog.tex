\documentclass[12pt,a4paper]{article}
\usepackage[slovak]{babel}
\usepackage[utf8]{inputenc}
\usepackage[IL2]{fontenc}
\usepackage{verbatim}
\usepackage{a4wide}
\title{\bf Dokumentácia ku knižnici \texttt{hpa}}
\date{\bf\today}
\author{\bf Peter Zeman}
\begin{document}
\maketitle

Knižnica {\tt hpa} implementuje sčitovanie, odčitovanie, násobenie, delenie a
mocnenie celých čísel, je naprogramovaná v {\tt C++}, je využité preťažovanie
operátorov.

Typ {\tt hp\_int} je definováný takto:

\begin{verbatim}

class hp_uint {
  usint base;
  vector<usint> digits
public:
  // verejne metody
};

class hp_int {
  sign_t sing;
  hp_uint absv;
public:
  // verejne metody, ktore zahrnaju operatory
};
\end{verbatim}

Typ {\tt hp\_int} pracuje s číslom v sústave so základom $2^{16}$. Všetky
algoritmy sú implementované pre {\tt hp\_uint} a {\tt hp\_int} doplňuje prácu
so znamienkom.

Pre sčitovanie a odčitovanie sú použité klasické algoritmy. Pre delenie
knižnica využíva rekurzívny algoritmus z [2], pre násobenie je implemenotvaný
základný a Karatsubov algoritmus. Na mocnenie je použité zrýchlenie pomocou
polenia intervalov.

\begin{thebibliography}{9}
	\bibitem{knuth}{\em Donald E. Knuth:}
	       {\bf The Art of Computer Programming, Volume II: Seminumerical
	       algorithms}\\
	       Addison-Wesley
	       1998
        \bibitem{algo}{\em S. Dasgupta, C. H. Papadimitriou, U. V. Vazirani:}
		{\bf Algorithms}\\
		2006
\end{thebibliography}

\end{document}
