\documentclass[12pt,a4paper]{article}
\usepackage[slovak]{babel}
\usepackage[utf8]{inputenc}
\usepackage[IL2]{fontenc}
\usepackage{verbatim}
\usepackage{a4wide}
\title{\bf Dokumentácia ku knižnici \texttt{hpa}}
\date{\bf\today}
\author{\bf Peter Zeman}
\begin{document}
\maketitle

Knižnica {\tt hpa} implementuje sčitovanie, odčitovanie, násobenie, delenie a
mocnenie celých čísel, je naprogramovaná v {\tt C++}, je využité preťažovanie
operátorov.

\subsection*{Použitie}

Na začiatku každého zdrojového súboru je treba uviesť

\begin{verbatim}
#include "hpa.h"
using namespace hpa;
\end{verbatim}

\noindent Príkazom

\begin{verbatim}
hp_int a;
\end{verbatim}

\noindent je možné deklarovať premennú {\tt a} typu {\tt hp\_int}. V tomto
prípade bude {\tt a} rovné nule. Premennú je možné inicializovať príkazom

\begin{verbatim}
hp_int a("123456789");
\end{verbatim}

\noindent ktorý iniclializje {\tt a} zo {\tt std::string}u.

\noindent Je možné použiť relačné operátory {\tt ==}, {\tt !=}, {\tt <}, {\tt
>}, {\tt <=} {\tt >=}, aritmetické operátory {\tt +}, {\tt -}, {\tt *}, {\tt
/}, {\tt \%}, {\tt \textasciicircum} (tento operátor slúži na mocnenie, pozor kôli priorite
je nutné ho uzátvorkovať, teda správny výraz je {\tt (a\textasciicircum b)} a
nie {\tt a\textasciicircum b})
a knim prislúchajúce priraďovacie operátory {\tt +=}, {\tt -=}, {\tt *=}, {\tt
/=}, {\tt \%=}, {\tt \textasciicircum=}.

\noindent Pre vstup a výstup je možné použiť operátory {\tt >>} a {\tt <<}:

\begin{verbatim}
hp_uint a("123442348729837409213470123");
cout << a;
hp_uint b;
cin >> b;
\end{verbatim}

\noindent Príkaz

\begin{verbatim}
a.set_base(b);
\end{verbatim}

\noindent prestaví sústavu, v ktorej bude očakávaný vstup a v ktore bude
zapísaný výstup.

\end{document}
